% Options for packages loaded elsewhere
\PassOptionsToPackage{unicode}{hyperref}
\PassOptionsToPackage{hyphens}{url}
%
\documentclass[
]{article}
\usepackage{amsmath,amssymb}
\usepackage{iftex}
\ifPDFTeX
  \usepackage[T1]{fontenc}
  \usepackage[utf8]{inputenc}
  \usepackage{textcomp} % provide euro and other symbols
\else % if luatex or xetex
  \usepackage{unicode-math} % this also loads fontspec
  \defaultfontfeatures{Scale=MatchLowercase}
  \defaultfontfeatures[\rmfamily]{Ligatures=TeX,Scale=1}
\fi
\usepackage{lmodern}
\ifPDFTeX\else
  % xetex/luatex font selection
\fi
% Use upquote if available, for straight quotes in verbatim environments
\IfFileExists{upquote.sty}{\usepackage{upquote}}{}
\IfFileExists{microtype.sty}{% use microtype if available
  \usepackage[]{microtype}
  \UseMicrotypeSet[protrusion]{basicmath} % disable protrusion for tt fonts
}{}
\makeatletter
\@ifundefined{KOMAClassName}{% if non-KOMA class
  \IfFileExists{parskip.sty}{%
    \usepackage{parskip}
  }{% else
    \setlength{\parindent}{0pt}
    \setlength{\parskip}{6pt plus 2pt minus 1pt}}
}{% if KOMA class
  \KOMAoptions{parskip=half}}
\makeatother
\usepackage{xcolor}
\usepackage[margin=1in]{geometry}
\usepackage{color}
\usepackage{fancyvrb}
\newcommand{\VerbBar}{|}
\newcommand{\VERB}{\Verb[commandchars=\\\{\}]}
\DefineVerbatimEnvironment{Highlighting}{Verbatim}{commandchars=\\\{\}}
% Add ',fontsize=\small' for more characters per line
\usepackage{framed}
\definecolor{shadecolor}{RGB}{248,248,248}
\newenvironment{Shaded}{\begin{snugshade}}{\end{snugshade}}
\newcommand{\AlertTok}[1]{\textcolor[rgb]{0.94,0.16,0.16}{#1}}
\newcommand{\AnnotationTok}[1]{\textcolor[rgb]{0.56,0.35,0.01}{\textbf{\textit{#1}}}}
\newcommand{\AttributeTok}[1]{\textcolor[rgb]{0.13,0.29,0.53}{#1}}
\newcommand{\BaseNTok}[1]{\textcolor[rgb]{0.00,0.00,0.81}{#1}}
\newcommand{\BuiltInTok}[1]{#1}
\newcommand{\CharTok}[1]{\textcolor[rgb]{0.31,0.60,0.02}{#1}}
\newcommand{\CommentTok}[1]{\textcolor[rgb]{0.56,0.35,0.01}{\textit{#1}}}
\newcommand{\CommentVarTok}[1]{\textcolor[rgb]{0.56,0.35,0.01}{\textbf{\textit{#1}}}}
\newcommand{\ConstantTok}[1]{\textcolor[rgb]{0.56,0.35,0.01}{#1}}
\newcommand{\ControlFlowTok}[1]{\textcolor[rgb]{0.13,0.29,0.53}{\textbf{#1}}}
\newcommand{\DataTypeTok}[1]{\textcolor[rgb]{0.13,0.29,0.53}{#1}}
\newcommand{\DecValTok}[1]{\textcolor[rgb]{0.00,0.00,0.81}{#1}}
\newcommand{\DocumentationTok}[1]{\textcolor[rgb]{0.56,0.35,0.01}{\textbf{\textit{#1}}}}
\newcommand{\ErrorTok}[1]{\textcolor[rgb]{0.64,0.00,0.00}{\textbf{#1}}}
\newcommand{\ExtensionTok}[1]{#1}
\newcommand{\FloatTok}[1]{\textcolor[rgb]{0.00,0.00,0.81}{#1}}
\newcommand{\FunctionTok}[1]{\textcolor[rgb]{0.13,0.29,0.53}{\textbf{#1}}}
\newcommand{\ImportTok}[1]{#1}
\newcommand{\InformationTok}[1]{\textcolor[rgb]{0.56,0.35,0.01}{\textbf{\textit{#1}}}}
\newcommand{\KeywordTok}[1]{\textcolor[rgb]{0.13,0.29,0.53}{\textbf{#1}}}
\newcommand{\NormalTok}[1]{#1}
\newcommand{\OperatorTok}[1]{\textcolor[rgb]{0.81,0.36,0.00}{\textbf{#1}}}
\newcommand{\OtherTok}[1]{\textcolor[rgb]{0.56,0.35,0.01}{#1}}
\newcommand{\PreprocessorTok}[1]{\textcolor[rgb]{0.56,0.35,0.01}{\textit{#1}}}
\newcommand{\RegionMarkerTok}[1]{#1}
\newcommand{\SpecialCharTok}[1]{\textcolor[rgb]{0.81,0.36,0.00}{\textbf{#1}}}
\newcommand{\SpecialStringTok}[1]{\textcolor[rgb]{0.31,0.60,0.02}{#1}}
\newcommand{\StringTok}[1]{\textcolor[rgb]{0.31,0.60,0.02}{#1}}
\newcommand{\VariableTok}[1]{\textcolor[rgb]{0.00,0.00,0.00}{#1}}
\newcommand{\VerbatimStringTok}[1]{\textcolor[rgb]{0.31,0.60,0.02}{#1}}
\newcommand{\WarningTok}[1]{\textcolor[rgb]{0.56,0.35,0.01}{\textbf{\textit{#1}}}}
\usepackage{graphicx}
\makeatletter
\def\maxwidth{\ifdim\Gin@nat@width>\linewidth\linewidth\else\Gin@nat@width\fi}
\def\maxheight{\ifdim\Gin@nat@height>\textheight\textheight\else\Gin@nat@height\fi}
\makeatother
% Scale images if necessary, so that they will not overflow the page
% margins by default, and it is still possible to overwrite the defaults
% using explicit options in \includegraphics[width, height, ...]{}
\setkeys{Gin}{width=\maxwidth,height=\maxheight,keepaspectratio}
% Set default figure placement to htbp
\makeatletter
\def\fps@figure{htbp}
\makeatother
\setlength{\emergencystretch}{3em} % prevent overfull lines
\providecommand{\tightlist}{%
  \setlength{\itemsep}{0pt}\setlength{\parskip}{0pt}}
\setcounter{secnumdepth}{-\maxdimen} % remove section numbering
\ifLuaTeX
  \usepackage{selnolig}  % disable illegal ligatures
\fi
\IfFileExists{bookmark.sty}{\usepackage{bookmark}}{\usepackage{hyperref}}
\IfFileExists{xurl.sty}{\usepackage{xurl}}{} % add URL line breaks if available
\urlstyle{same}
\hypersetup{
  pdftitle={Challenge-3},
  pdfauthor={Tang Ching Xian},
  hidelinks,
  pdfcreator={LaTeX via pandoc}}

\title{Challenge-3}
\author{Tang Ching Xian}
\date{27/08/2023}

\begin{document}
\maketitle

\hypertarget{i.-questions}{%
\subsection{I. Questions}\label{i.-questions}}

\hypertarget{question-1-emoji-expressions}{%
\paragraph{Question 1: Emoji
Expressions}\label{question-1-emoji-expressions}}

Imagine you're analyzing social media posts for sentiment analysis. If
you were to create a variable named ``postSentiment'' to store the
sentiment of a post using emojis (😄 for positive, 😐 for neutral, 😢
for negative), what data type would you assign to this variable? Why?
(\emph{narrative type question, no code required})

\textbf{Solution:} Since the emojis represent positive, neutral and
negative, they represents catergories with order, thus they are ordinal
categoric data which are normally character. Since we are using
``postSentiment'', it would likely to be assigned the data type of
``string''.

\hypertarget{question-2-hashtag-havoc}{%
\paragraph{Question 2: Hashtag Havoc}\label{question-2-hashtag-havoc}}

In a study on trending hashtags, you want to store the list of hashtags
associated with a post. What data type would you choose for the variable
``postHashtags''? How might this data type help you analyze and
categorize the hashtags later? (\emph{narrative type question, no code
required})

\textbf{Solution:} postHashtags would likely to be datatype ``list'' or
character string. Element in a list can be named and list are useful for
storing hashtags since they can vary in number and their order of
appearance may be important. Therefore, with list, one can categories
the data based on criteria and iterate through the list, tally up the
occurrence of each hashtag and identify trends and patterns.

\hypertarget{question-3-time-travelers-log}{%
\paragraph{Question 3: Time Traveler's
Log}\label{question-3-time-travelers-log}}

You're examining the timing of user interactions on a website. Would you
use a numeric or non-numeric data type to represent the timestamp of
each interaction? Explain your choice (\emph{narrative type question, no
code required})

\textbf{Solution:} I would use non-numeric data type since
``-'',month=``april'',2.2.2023 all these would be classified as string
and continuous double therefore non-numeric.

\hypertarget{question-4-event-elegance}{%
\paragraph{Question 4: Event Elegance}\label{question-4-event-elegance}}

You're managing an event database that includes the date and time of
each session. What data type(s) would you use to represent the session
date and time? (\emph{narrative type question, no code required})

\textbf{Solution:} I would use a continuous numeric datatype because
time and date are numerical values, it is continuous and has order of
progression of time. Using numeric datatype, one will be able to store
and manipulate time-related data easily.

\hypertarget{question-5-nominee-nominations}{%
\paragraph{Question 5: Nominee
Nominations}\label{question-5-nominee-nominations}}

You're analyzing nominations for an online award. Each participant can
nominate multiple candidates. What data type would be suitable for
storing the list of nominated candidates for each participant?
(\emph{narrative type question, no code required})

\textbf{Solution:} Since each participants can nominate multiple
candidates, using list data type can elicit names in the list, retrieve
the element specifically named and truncated list.

\hypertarget{question-6-communication-channels}{%
\paragraph{Question 6: Communication
Channels}\label{question-6-communication-channels}}

In a survey about preferred communication channels, respondents choose
from options like ``email,'' ``phone,'' or ``social media.'' What data
type would you assign to the variable ``preferredChannel''?
(\emph{narrative type question, no code required})

\textbf{Solution:} I woudl assign categoric data type for
``preferredChannel'' since categorical dat type is used to represent
data that has distinct categories without any inherent order. Since
``email'',``phone'', ``social media'' are discrete, ``preferredChannel''
variable is suitable because it accurately reflects the data without
implying any sort of numeric relationship.

\hypertarget{question-7-colorful-commentary}{%
\paragraph{Question 7: Colorful
Commentary}\label{question-7-colorful-commentary}}

In a design feedback survey, participants are asked to describe their
feelings about a website using color names (e.g., ``warm red,'' ``cool
blue''). What data type would you choose for the variable
``feedbackColor''? (\emph{narrative type question, no code required})

\textbf{Solution:} Since the color names are ``red'' and ``blue'', they
are Categoric variables and they are nominal as they do not have natural
ordering. Therefore their type is character or logical. But in this
case, it should be string of characters.

\hypertarget{question-8-variable-exploration}{%
\paragraph{Question 8: Variable
Exploration}\label{question-8-variable-exploration}}

Imagine you're conducting a study on social media usage. Identify three
variables related to this study, and specify their data types in R.
Classify each variable as either numeric or non-numeric.

\textbf{Solution:}\\
Variable 1: Number of followers Data type: Numeric (integer; whole
number)

Variable 2: Time spent on social media Data type: Numeric (Double;
decimal values)

Variable 3: Favourite social media platform Data type: Catgoric
character (String; non-numeric)

\hypertarget{question-9-vector-variety}{%
\paragraph{Question 9: Vector Variety}\label{question-9-vector-variety}}

Create a numeric vector named ``ages'' containing the ages of five
people: 25, 30, 22, 28, and 33. Print the vector.

\textbf{Solution:}

\begin{Shaded}
\begin{Highlighting}[]
\NormalTok{ages}\OtherTok{\textless{}{-}} \FunctionTok{c}\NormalTok{(25L,30L,22L,28L,33L)}
\FunctionTok{print}\NormalTok{(ages)}
\end{Highlighting}
\end{Shaded}

\begin{verbatim}
## [1] 25 30 22 28 33
\end{verbatim}

\hypertarget{question-10-list-logic}{%
\paragraph{Question 10: List Logic}\label{question-10-list-logic}}

Construct a list named ``student\_info'' that contains the following
elements:

\begin{itemize}
\item
  A character vector of student names: ``Alice,'' ``Bob,'' ``Catherine''
\item
  A numeric vector of their respective scores: 85, 92, 78
\item
  A logical vector indicating if they passed the exam: TRUE, TRUE, FALSE
\end{itemize}

Print the list.

\textbf{Solution:}

\begin{Shaded}
\begin{Highlighting}[]
\NormalTok{student\_info}\OtherTok{\textless{}{-}}\FunctionTok{list}\NormalTok{ (}\AttributeTok{name=}\FunctionTok{c}\NormalTok{(}\StringTok{"Alice"}\NormalTok{,}\StringTok{"Bob"}\NormalTok{,}\StringTok{"Catherine"}\NormalTok{), }\AttributeTok{scores=}\FunctionTok{c}\NormalTok{(}\DecValTok{85}\NormalTok{,}\DecValTok{92}\NormalTok{,}\DecValTok{78}\NormalTok{),}\AttributeTok{passed\_exam=}\FunctionTok{c}\NormalTok{(}\ConstantTok{TRUE}\NormalTok{,}\ConstantTok{TRUE}\NormalTok{,}\ConstantTok{FALSE}\NormalTok{))}
\FunctionTok{print}\NormalTok{(student\_info)}
\end{Highlighting}
\end{Shaded}

\begin{verbatim}
## $name
## [1] "Alice"     "Bob"       "Catherine"
## 
## $scores
## [1] 85 92 78
## 
## $passed_exam
## [1]  TRUE  TRUE FALSE
\end{verbatim}

\hypertarget{question-11-type-tracking}{%
\paragraph{Question 11: Type Tracking}\label{question-11-type-tracking}}

You have a vector ``data'' containing the values 10, 15.5, ``20'', and
TRUE. Determine the data types of each element using the typeof()
function.

\textbf{Solution:}

\begin{Shaded}
\begin{Highlighting}[]
\CommentTok{\# Create a vector}
\NormalTok{x }\OtherTok{\textless{}{-}} \FunctionTok{c}\NormalTok{(}\DecValTok{10}\NormalTok{,}\FloatTok{15.5}\NormalTok{,}\StringTok{"20"}\NormalTok{,}\ConstantTok{TRUE}\NormalTok{)}
\CommentTok{\# Check the type of x}
\FunctionTok{typeof}\NormalTok{(x[}\DecValTok{1}\NormalTok{])}
\end{Highlighting}
\end{Shaded}

\begin{verbatim}
## [1] "character"
\end{verbatim}

\begin{Shaded}
\begin{Highlighting}[]
\FunctionTok{typeof}\NormalTok{(x[}\DecValTok{2}\NormalTok{])}
\end{Highlighting}
\end{Shaded}

\begin{verbatim}
## [1] "character"
\end{verbatim}

\begin{Shaded}
\begin{Highlighting}[]
\FunctionTok{typeof}\NormalTok{(x[}\DecValTok{3}\NormalTok{])}
\end{Highlighting}
\end{Shaded}

\begin{verbatim}
## [1] "character"
\end{verbatim}

\begin{Shaded}
\begin{Highlighting}[]
\FunctionTok{typeof}\NormalTok{(x[}\DecValTok{4}\NormalTok{])}
\end{Highlighting}
\end{Shaded}

\begin{verbatim}
## [1] "character"
\end{verbatim}

\hypertarget{question-12-coercion-chronicles}{%
\paragraph{Question 12: Coercion
Chronicles}\label{question-12-coercion-chronicles}}

You have a numeric vector ``prices'' with values 20.5, 15, and ``25''.
Use explicit coercion to convert the last element to a numeric data
type. Print the updated vector.

\textbf{Solution:}

\begin{Shaded}
\begin{Highlighting}[]
\NormalTok{prices }\OtherTok{\textless{}{-}} \FunctionTok{c}\NormalTok{(}\FloatTok{20.5}\NormalTok{, }\DecValTok{15}\NormalTok{, }\StringTok{"25"}\NormalTok{)}

\CommentTok{\# Convert the last element to a numeric data type}
\NormalTok{prices[}\DecValTok{3}\NormalTok{] }\OtherTok{\textless{}{-}} \FunctionTok{as.numeric}\NormalTok{(prices[}\DecValTok{3}\NormalTok{])}
\end{Highlighting}
\end{Shaded}

\hypertarget{question-13-implicit-intuition}{%
\paragraph{Question 13: Implicit
Intuition}\label{question-13-implicit-intuition}}

Combine the numeric vector c(5, 10, 15) with the character vector
c(``apple'', ``banana'', ``cherry''). What happens to the data types of
the combined vector? Explain the concept of implicit coercion.

\textbf{Solution:}

\begin{Shaded}
\begin{Highlighting}[]
\NormalTok{numeric\_vector }\OtherTok{\textless{}{-}} \FunctionTok{c}\NormalTok{(}\DecValTok{5}\NormalTok{, }\DecValTok{10}\NormalTok{, }\DecValTok{15}\NormalTok{)}
\NormalTok{character\_vector }\OtherTok{\textless{}{-}} \FunctionTok{c}\NormalTok{(}\StringTok{"apple"}\NormalTok{, }\StringTok{"banana"}\NormalTok{, }\StringTok{"cherry"}\NormalTok{)}
\NormalTok{combined\_vector }\OtherTok{\textless{}{-}} \FunctionTok{c}\NormalTok{(numeric\_vector, character\_vector)}

\FunctionTok{print}\NormalTok{(combined\_vector)}
\end{Highlighting}
\end{Shaded}

\begin{verbatim}
## [1] "5"      "10"     "15"     "apple"  "banana" "cherry"
\end{verbatim}

\begin{Shaded}
\begin{Highlighting}[]
\FunctionTok{print}\NormalTok{(}\FunctionTok{typeof}\NormalTok{(combined\_vector))}
\end{Highlighting}
\end{Shaded}

\begin{verbatim}
## [1] "character"
\end{verbatim}

The data types of the combined vector will be implicitly coerced to a
common data type. In this case, the resulting vector will be a
characater data type due to the presence of character vector, hence R
convert the numeric data type to character data type based on its
content.

\hypertarget{question-14-coercion-challenges}{%
\paragraph{Question 14: Coercion
Challenges}\label{question-14-coercion-challenges}}

You have a vector ``numbers'' with values 7, 12.5, and ``15.7''.
Calculate the sum of these numbers. Will R automatically handle the data
type conversion? If not, how would you handle it?

\textbf{Solution:} Mixing different data types (numeric and character)
in arithmetic operations, R will not automatically handle the data type
conversion. Since R will not be able to automatically handle the data
type conversion, I need to use explicit coercion to convert the
character into numeric using as.numeric.

\begin{Shaded}
\begin{Highlighting}[]
\NormalTok{ numbers }\OtherTok{\textless{}{-}} \FunctionTok{c}\NormalTok{(}\DecValTok{7}\NormalTok{, }\FloatTok{12.5}\NormalTok{, }\StringTok{"15.7"}\NormalTok{)}

\CommentTok{\# Convert character elements to numeric}
\NormalTok{numbers }\OtherTok{\textless{}{-}} \FunctionTok{as.numeric}\NormalTok{(numbers)}

\CommentTok{\# Calculate the sum}
\NormalTok{sum\_result }\OtherTok{\textless{}{-}} \FunctionTok{sum}\NormalTok{(numbers)}

\CommentTok{\#simplify }
\CommentTok{\#sum(as.numeric(numbers))}

\FunctionTok{print}\NormalTok{(sum\_result)}
\end{Highlighting}
\end{Shaded}

\begin{verbatim}
## [1] 35.2
\end{verbatim}

\hypertarget{question-15-coercion-consequences}{%
\paragraph{Question 15: Coercion
Consequences}\label{question-15-coercion-consequences}}

Suppose you want to calculate the average of a vector ``grades'' with
values 85, 90.5, and ``75.2''. If you directly calculate the mean using
the mean() function, what result do you expect? How might you ensure
accurate calculation?

\textbf{Solution:} If I directly calculate the mean of the vector
``grades'' with values 85, 90.5, and ``75.2'' using the mean() function
without any data type conversion,can lead to incorrect results such as
85. As shown here, mean(85,90.5,``75.2'') {[}1{]} 85 OR grades
\textless- c(85, 90.5, ``75.2'') mean(grades) {[}1{]} NA Mean() function
requires the input vector to contain only numeric values.

\begin{Shaded}
\begin{Highlighting}[]
\NormalTok{grades }\OtherTok{\textless{}{-}} \FunctionTok{c}\NormalTok{(}\DecValTok{85}\NormalTok{, }\FloatTok{90.5}\NormalTok{, }\StringTok{"75.2"}\NormalTok{)}

\CommentTok{\# Convert character element to numeric}
\NormalTok{grades }\OtherTok{\textless{}{-}} \FunctionTok{as.numeric}\NormalTok{(grades)}

\CommentTok{\# Calculate the mean}
\NormalTok{mean\_result }\OtherTok{\textless{}{-}} \FunctionTok{mean}\NormalTok{(grades)}

\CommentTok{\# Simplify}
\CommentTok{\#mean\_result \textless{}{-} mean(as.numeric(grades))}
\FunctionTok{print}\NormalTok{(mean\_result)}
\end{Highlighting}
\end{Shaded}

\begin{verbatim}
## [1] 83.56667
\end{verbatim}

\hypertarget{question-16-data-diversity-in-lists}{%
\paragraph{Question 16: Data Diversity in
Lists}\label{question-16-data-diversity-in-lists}}

Create a list named ``mixed\_data'' with the following components:

\begin{itemize}
\item
  A numeric vector: 10, 20, 30
\item
  A character vector: ``red'', ``green'', ``blue''
\item
  A logical vector: TRUE, FALSE, TRUE
\end{itemize}

Calculate the mean of the numeric vector within the list.

\textbf{Solution:}

\begin{Shaded}
\begin{Highlighting}[]
\CommentTok{\# Creating the list with different components}
\NormalTok{mixed\_data }\OtherTok{\textless{}{-}} \FunctionTok{list}\NormalTok{(}\AttributeTok{numeric\_vector =} \FunctionTok{c}\NormalTok{(}\DecValTok{10}\NormalTok{, }\DecValTok{20}\NormalTok{, }\DecValTok{30}\NormalTok{),}\AttributeTok{character\_vector =} \FunctionTok{c}\NormalTok{(}\StringTok{"red"}\NormalTok{, }\StringTok{"green"}\NormalTok{, }\StringTok{"blue"}\NormalTok{),}\AttributeTok{logical\_vector =} \FunctionTok{c}\NormalTok{(}\ConstantTok{TRUE}\NormalTok{, }\ConstantTok{FALSE}\NormalTok{, }\ConstantTok{TRUE}\NormalTok{))}

\CommentTok{\# Calculate the mean of the numeric vector within the list}
\NormalTok{mean\_numeric }\OtherTok{\textless{}{-}} \FunctionTok{mean}\NormalTok{(mixed\_data}\SpecialCharTok{$}\NormalTok{numeric\_vector)}
\CommentTok{\#Another way to calculate the mean of the numeric vector within the list}
\CommentTok{\#mean\_numeric \textless{}{-} mean(mixed\_data[["numeric\_vector"]])}
\FunctionTok{print}\NormalTok{(mean\_numeric)}
\end{Highlighting}
\end{Shaded}

\begin{verbatim}
## [1] 20
\end{verbatim}

\hypertarget{question-17-list-logic-follow-up}{%
\paragraph{Question 17: List Logic
Follow-up}\label{question-17-list-logic-follow-up}}

Using the ``student\_info'' list from Question 10, extract and print the
score of the student named ``Bob.''

\textbf{Solution:}

\begin{Shaded}
\begin{Highlighting}[]
\CommentTok{\# Assuming you have the "student\_info" list defined from Question 10}
\NormalTok{student\_info}\OtherTok{\textless{}{-}}\FunctionTok{list}\NormalTok{(}\AttributeTok{names =} \FunctionTok{c}\NormalTok{(}\StringTok{"Alice"}\NormalTok{, }\StringTok{"Bob"}\NormalTok{, }\StringTok{"Catherine"}\NormalTok{),}\AttributeTok{scores =} \FunctionTok{c}\NormalTok{(}\DecValTok{85}\NormalTok{, }\DecValTok{92}\NormalTok{, }\DecValTok{78}\NormalTok{),}\AttributeTok{passed\_exam =} \FunctionTok{c}\NormalTok{(}\ConstantTok{TRUE}\NormalTok{, }\ConstantTok{TRUE}\NormalTok{, }\ConstantTok{FALSE}\NormalTok{))}

\CommentTok{\# Extract and print the score of the student named "Bob"}
\NormalTok{bob\_score}\OtherTok{\textless{}{-}}\NormalTok{student\_info}\SpecialCharTok{$}\NormalTok{scores[student\_info}\SpecialCharTok{$}\NormalTok{names }\SpecialCharTok{==} \StringTok{"Bob"}\NormalTok{]}

\FunctionTok{print}\NormalTok{(bob\_score)}
\end{Highlighting}
\end{Shaded}

\begin{verbatim}
## [1] 92
\end{verbatim}

\hypertarget{question-18-dynamic-access}{%
\paragraph{Question 18: Dynamic
Access}\label{question-18-dynamic-access}}

Create a numeric vector values with random values. Write R code to
dynamically access and print the last element of the vector, regardless
of its length.

\textbf{Solution:}

\begin{Shaded}
\begin{Highlighting}[]
\CommentTok{\# Create a numeric vector with random random values}
\CommentTok{\# values \textless{}{-} runif(10, min = 0, max = 100)}

\CommentTok{\# Create a numeric vector with random fixed values}
\NormalTok{values }\OtherTok{\textless{}{-}} \FunctionTok{c}\NormalTok{(}\DecValTok{10}\NormalTok{, }\DecValTok{20}\NormalTok{, }\DecValTok{30}\NormalTok{, }\DecValTok{40}\NormalTok{, }\DecValTok{50}\NormalTok{)}

\CommentTok{\# Get the length of the vector}
\NormalTok{vector\_length }\OtherTok{\textless{}{-}} \FunctionTok{length}\NormalTok{(values)}

\CommentTok{\# Access and print the last element dynamically}
\NormalTok{last\_element }\OtherTok{\textless{}{-}}\NormalTok{ values[vector\_length]}
\FunctionTok{print}\NormalTok{(last\_element)}
\end{Highlighting}
\end{Shaded}

\begin{verbatim}
## [1] 50
\end{verbatim}

\begin{Shaded}
\begin{Highlighting}[]
\DocumentationTok{\#\#or values[length(values)]}
\end{Highlighting}
\end{Shaded}

\hypertarget{question-19-multiple-matches}{%
\paragraph{Question 19: Multiple
Matches}\label{question-19-multiple-matches}}

You have a character vector words \textless- c(``apple'', ``banana'',
``cherry'', ``apple''). Write R code to find and print the indices of
all occurrences of the word ``apple.''

\textbf{Solution:}

\begin{Shaded}
\begin{Highlighting}[]
\NormalTok{words }\OtherTok{\textless{}{-}} \FunctionTok{c}\NormalTok{(}\StringTok{"apple"}\NormalTok{, }\StringTok{"banana"}\NormalTok{, }\StringTok{"cherry"}\NormalTok{, }\StringTok{"apple"}\NormalTok{)}

\CommentTok{\# Find indices of all occurrences of "apple"}
\NormalTok{apple\_indices }\OtherTok{\textless{}{-}} \FunctionTok{which}\NormalTok{(words }\SpecialCharTok{==} \StringTok{"apple"}\NormalTok{)}

\CommentTok{\# Print the indices\#}
\FunctionTok{print}\NormalTok{(apple\_indices)}
\end{Highlighting}
\end{Shaded}

\begin{verbatim}
## [1] 1 4
\end{verbatim}

\hypertarget{question-20-conditional-capture}{%
\paragraph{Question 20: Conditional
Capture}\label{question-20-conditional-capture}}

Assume you have a vector ages containing the ages of individuals. Write
R code to extract and print the ages of individuals who are older than
30.

\textbf{Solution:}

\begin{Shaded}
\begin{Highlighting}[]
\NormalTok{ages }\OtherTok{\textless{}{-}} \FunctionTok{c}\NormalTok{(}\DecValTok{25}\NormalTok{, }\DecValTok{45}\NormalTok{, }\DecValTok{32}\NormalTok{, }\DecValTok{28}\NormalTok{, }\DecValTok{55}\NormalTok{, }\DecValTok{40}\NormalTok{)}

\CommentTok{\# Extract ages of individuals older than 30}
\NormalTok{older\_than\_30 }\OtherTok{\textless{}{-}}\NormalTok{ ages[ages }\SpecialCharTok{\textgreater{}} \DecValTok{30}\NormalTok{]}

\CommentTok{\# Print the ages}
\FunctionTok{print}\NormalTok{(older\_than\_30)}
\end{Highlighting}
\end{Shaded}

\begin{verbatim}
## [1] 45 32 55 40
\end{verbatim}

\hypertarget{question-21-extract-every-nth}{%
\paragraph{Question 21: Extract Every
Nth}\label{question-21-extract-every-nth}}

Given a numeric vector sequence \textless- 1:20, write R code to extract
and print every third element of the vector.

\textbf{Solution:}

\begin{Shaded}
\begin{Highlighting}[]
\CommentTok{\# Numeric vector from 1 to 20}
\NormalTok{sequence}\OtherTok{\textless{}{-}}\DecValTok{1}\SpecialCharTok{:}\DecValTok{20}

\CommentTok{\# Extract and print every third element}
\NormalTok{every\_third }\OtherTok{\textless{}{-}}\NormalTok{ sequence[}\FunctionTok{seq}\NormalTok{(}\AttributeTok{from =} \DecValTok{3}\NormalTok{, }\AttributeTok{to =} \FunctionTok{length}\NormalTok{(sequence), }\AttributeTok{by =} \DecValTok{3}\NormalTok{)]}

\CommentTok{\# Print the extracted elements}
\FunctionTok{print}\NormalTok{(every\_third)}
\end{Highlighting}
\end{Shaded}

\begin{verbatim}
## [1]  3  6  9 12 15 18
\end{verbatim}

\hypertarget{question-22-range-retrieval}{%
\paragraph{Question 22: Range
Retrieval}\label{question-22-range-retrieval}}

Create a numeric vector numbers with values from 1 to 10. Write R code
to extract and print the values between the fourth and eighth elements.

\textbf{Solution:}

\begin{Shaded}
\begin{Highlighting}[]
\CommentTok{\# Create a numeric vector from 1 to 10}
\NormalTok{numbers }\OtherTok{\textless{}{-}} \DecValTok{1}\SpecialCharTok{:}\DecValTok{10}

\CommentTok{\# Extract and print values between the fourth and eighth elements}
\NormalTok{between\_fourth\_and\_eighth }\OtherTok{\textless{}{-}}\NormalTok{ numbers[}\DecValTok{4}\SpecialCharTok{:}\DecValTok{8}\NormalTok{]}

\CommentTok{\# Print the extracted values}
\FunctionTok{print}\NormalTok{(between\_fourth\_and\_eighth)}
\end{Highlighting}
\end{Shaded}

\begin{verbatim}
## [1] 4 5 6 7 8
\end{verbatim}

\hypertarget{question-23-missing-matters}{%
\paragraph{Question 23: Missing
Matters}\label{question-23-missing-matters}}

Suppose you have a numeric vector data \textless- c(10, NA, 15, 20).
Write R code to check if the second element of the vector is missing
(NA).

\textbf{Solution:}

\begin{Shaded}
\begin{Highlighting}[]
\CommentTok{\# Numeric vector with NA}
\NormalTok{data }\OtherTok{\textless{}{-}} \FunctionTok{c}\NormalTok{(}\DecValTok{10}\NormalTok{, }\ConstantTok{NA}\NormalTok{, }\DecValTok{15}\NormalTok{, }\DecValTok{20}\NormalTok{)}
\NormalTok{data[}\DecValTok{2}\NormalTok{]}
\end{Highlighting}
\end{Shaded}

\begin{verbatim}
## [1] NA
\end{verbatim}

\begin{Shaded}
\begin{Highlighting}[]
\CommentTok{\# Check if the second element is NA}
\NormalTok{is\_second\_element\_missing }\OtherTok{\textless{}{-}} \FunctionTok{is.na}\NormalTok{(data[}\DecValTok{2}\NormalTok{])}

\CommentTok{\# Print the result}
\FunctionTok{print}\NormalTok{(is\_second\_element\_missing)}
\end{Highlighting}
\end{Shaded}

\begin{verbatim}
## [1] TRUE
\end{verbatim}

\hypertarget{question-24-temperature-extremes}{%
\paragraph{Question 24: Temperature
Extremes}\label{question-24-temperature-extremes}}

Assume you have a numeric vector temperatures with daily temperatures.
Create a logical vector hot\_days that flags days with temperatures
above 90 degrees Fahrenheit. Print the total number of hot days.

\textbf{Solution:}

\begin{Shaded}
\begin{Highlighting}[]
\CommentTok{\# Numeric vector of daily temperatures}
\NormalTok{temperatures }\OtherTok{\textless{}{-}} \FunctionTok{c}\NormalTok{(}\DecValTok{88}\NormalTok{, }\DecValTok{92}\NormalTok{, }\DecValTok{87}\NormalTok{, }\DecValTok{95}\NormalTok{, }\DecValTok{89}\NormalTok{, }\DecValTok{92}\NormalTok{, }\DecValTok{91}\NormalTok{, }\DecValTok{86}\NormalTok{, }\DecValTok{93}\NormalTok{, }\DecValTok{97}\NormalTok{, }\DecValTok{88}\NormalTok{, }\DecValTok{99}\NormalTok{, }\DecValTok{84}\NormalTok{, }\DecValTok{88}\NormalTok{, }\DecValTok{92}\NormalTok{)}

\CommentTok{\# Create a logical vector to flag hot days}
\NormalTok{hot\_days }\OtherTok{\textless{}{-}}\NormalTok{ temperatures }\SpecialCharTok{\textgreater{}} \DecValTok{90}

\CommentTok{\# Count the total number of hot days}
\NormalTok{total\_hot\_days }\OtherTok{\textless{}{-}} \FunctionTok{sum}\NormalTok{(hot\_days, }\AttributeTok{na.rm =} \ConstantTok{TRUE}\NormalTok{)}

\CommentTok{\# Print the total number of hot days}
\FunctionTok{print}\NormalTok{(total\_hot\_days)}
\end{Highlighting}
\end{Shaded}

\begin{verbatim}
## [1] 8
\end{verbatim}

\begin{Shaded}
\begin{Highlighting}[]
\CommentTok{\#another method}
\NormalTok{daily\_temp}\OtherTok{=}\FunctionTok{c}\NormalTok{(}\DecValTok{88}\NormalTok{, }\DecValTok{92}\NormalTok{, }\DecValTok{87}\NormalTok{, }\DecValTok{95}\NormalTok{, }\DecValTok{89}\NormalTok{, }\DecValTok{92}\NormalTok{, }\DecValTok{91}\NormalTok{, }\DecValTok{86}\NormalTok{, }\DecValTok{93}\NormalTok{, }\DecValTok{97}\NormalTok{, }\DecValTok{88}\NormalTok{, }\DecValTok{99}\NormalTok{, }\DecValTok{84}\NormalTok{, }\DecValTok{88}\NormalTok{, }\DecValTok{92}\NormalTok{)}
\NormalTok{hot\_days}\OtherTok{=}\FunctionTok{vector}\NormalTok{(}\StringTok{"logical"}\NormalTok{)}
\ControlFlowTok{for}\NormalTok{ (temp }\ControlFlowTok{in}\NormalTok{ daily\_temp)\{}\CommentTok{\#temp is all element}
  \ControlFlowTok{if}\NormalTok{ (temp}\SpecialCharTok{\textgreater{}}\DecValTok{90}\NormalTok{)\{}
\NormalTok{    hot\_days}\OtherTok{\textless{}{-}}\FunctionTok{c}\NormalTok{(hot\_days,}\ConstantTok{TRUE}\NormalTok{)\}}
  \ControlFlowTok{else}\NormalTok{\{}
\NormalTok{    hot\_days}\OtherTok{\textless{}{-}}\FunctionTok{c}\NormalTok{(hot\_days,}\ConstantTok{FALSE}\NormalTok{)\}}
\NormalTok{\}}
\FunctionTok{length}\NormalTok{(hot\_days[hot\_days}\SpecialCharTok{==}\ConstantTok{TRUE}\NormalTok{])}
\end{Highlighting}
\end{Shaded}

\begin{verbatim}
## [1] 8
\end{verbatim}

\begin{Shaded}
\begin{Highlighting}[]
\CommentTok{\#another method}
\NormalTok{hot\_days}\OtherTok{=} \FunctionTok{as.logical}\NormalTok{(daily\_temp[daily\_temp}\SpecialCharTok{\textgreater{}}\DecValTok{90}\NormalTok{])}
\FunctionTok{length}\NormalTok{(hot\_days)}
\end{Highlighting}
\end{Shaded}

\begin{verbatim}
## [1] 8
\end{verbatim}

\hypertarget{question-25-string-selection}{%
\paragraph{Question 25: String
Selection}\label{question-25-string-selection}}

Given a character vector fruits containing fruit names, create a logical
vector long\_names that identifies fruits with names longer than 6
characters. Print the long fruit names.

\textbf{Solution:}

\begin{Shaded}
\begin{Highlighting}[]
\CommentTok{\# Character vector of fruit names}
\NormalTok{fruits }\OtherTok{\textless{}{-}} \FunctionTok{c}\NormalTok{(}\StringTok{"apple"}\NormalTok{, }\StringTok{"banana"}\NormalTok{, }\StringTok{"strawberry"}\NormalTok{, }\StringTok{"kiwi"}\NormalTok{, }\StringTok{"blueberry"}\NormalTok{, }\StringTok{"orange"}\NormalTok{, }\StringTok{"grape"}\NormalTok{)}

\CommentTok{\# Create a logical vector to identify long fruit names}
\NormalTok{long\_names }\OtherTok{\textless{}{-}} \FunctionTok{nchar}\NormalTok{(fruits) }\SpecialCharTok{\textgreater{}} \DecValTok{6}

\CommentTok{\# Print the long fruit names}
\FunctionTok{print}\NormalTok{(fruits[long\_names])}
\end{Highlighting}
\end{Shaded}

\hypertarget{question-26-data-divisibility}{%
\paragraph{Question 26: Data
Divisibility}\label{question-26-data-divisibility}}

Given a numeric vector numbers, create a logical vector divisible\_by\_5
to indicate numbers that are divisible by 5. Print the numbers that
satisfy this condition.

\textbf{Solution:}

\begin{Shaded}
\begin{Highlighting}[]
\CommentTok{\# Numeric vector of numbers}
\NormalTok{numbers }\OtherTok{\textless{}{-}} \FunctionTok{c}\NormalTok{(}\DecValTok{10}\NormalTok{, }\DecValTok{25}\NormalTok{, }\DecValTok{14}\NormalTok{, }\DecValTok{15}\NormalTok{, }\DecValTok{30}\NormalTok{, }\DecValTok{8}\NormalTok{, }\DecValTok{20}\NormalTok{, }\DecValTok{16}\NormalTok{)}

\CommentTok{\# Create a logical vector to indicate numbers divisible by 5}
\NormalTok{divisible\_by\_5 }\OtherTok{\textless{}{-}}\NormalTok{ numbers }\SpecialCharTok{\%\%} \DecValTok{5} \SpecialCharTok{==} \DecValTok{0}

\CommentTok{\# Print the numbers that are divisible by 5}
\FunctionTok{print}\NormalTok{(numbers[divisible\_by\_5])}
\end{Highlighting}
\end{Shaded}

\hypertarget{question-27-bigger-or-smaller}{%
\paragraph{Question 27: Bigger or
Smaller?}\label{question-27-bigger-or-smaller}}

You have two numeric vectors vector1 and vector2. Create a logical
vector comparison to indicate whether each element in vector1 is greater
than the corresponding element in vector2. Print the comparison results.

\textbf{Solution:}

\begin{Shaded}
\begin{Highlighting}[]
\CommentTok{\# Sample numeric vectors}
\NormalTok{vector1 }\OtherTok{\textless{}{-}} \FunctionTok{c}\NormalTok{(}\DecValTok{10}\NormalTok{, }\DecValTok{25}\NormalTok{, }\DecValTok{14}\NormalTok{, }\DecValTok{15}\NormalTok{, }\DecValTok{30}\NormalTok{)}
\NormalTok{vector2 }\OtherTok{\textless{}{-}} \FunctionTok{c}\NormalTok{(}\DecValTok{8}\NormalTok{, }\DecValTok{20}\NormalTok{, }\DecValTok{12}\NormalTok{, }\DecValTok{10}\NormalTok{, }\DecValTok{25}\NormalTok{)}

\CommentTok{\# Create a logical vector for comparison}
\NormalTok{comparison }\OtherTok{\textless{}{-}}\NormalTok{ vector1 }\SpecialCharTok{\textgreater{}}\NormalTok{ vector2}

\CommentTok{\# Print the comparison results}
\FunctionTok{print}\NormalTok{(comparison)}
\end{Highlighting}
\end{Shaded}


\end{document}
